\section{Introduction}\label{sec:introduction}

Structure-from-Motion (SfM) is a vital technique in computer vision, enabling the reconstruction of 3D geometry from 2D image sequences. 
Its applications span diverse domains, including robotics, augmented reality, and photogrammetry. 
With a growing number of SfM algorithms, systematic evaluation becomes essential to assess their performance and applicability under varying scenarios.

This report builds upon a structured evaluation protocol to compare four contemporary SfM approaches: Glomap \cite{pan2024glomap}, VGGsfm \cite{wang2023vggsfm}, Flowmap \cite{smith24flowmap}, and Ace0 \cite{brachmann2024acezero}. 
Leveraging datasets such as ETH3D \cite{Schops_2019_CVPR}, Tanks and Temples \cite{Knapitsch2017}, and MipNeft360 \cite{barron2022mipnerf360}, 
this study benchmarks these algorithms using metrics such as camera pose error, 3D triangulation accuracy, novel view synthesis quality \cite{DBLP:journals/corr/WaechterBFMKG16} as well as time and memory efficiency.
These benchmarks aim to capture a holistic view of algorithm performance, considering both geometric accuracy and scene reconstruction fidelity.

Central to this evaluation is the computation of relative rotation error (RRE) and relative translation error (RTE) for assessing camera pose estimation. 
These metrics provide insights into the precision of relative camera transformations derived from SfM models. 
The evaluation protocol will be extended to incorporate further metrics, such as triangulation completeness and rendering accuracy for novel views.

This study aspires to provide actionable insights into the capabilities of modern SfM algorithms and guide their application across various datasets and domains. 
By dissecting their strengths and limitations, it aims to foster improvements in the design and deployment of SfM systems.