% arara: pdflatex: { synctex: yes }
% arara: makeindex: { style: ctuthesis }
% arara: bibtex

% The class takes all the key=value arguments that \ctusetup does,
% and a couple more: draft and oneside
\documentclass[oneside]{ctuthesis}

%— PDF compression settings (lossless) —%
\pdfminorversion=5
\pdfcompresslevel=9
\pdfobjcompresslevel=3

\ctusetup{
	preprint = \ctuverlog,
	mainlanguage = english,
%	titlelanguage = czech,
	% mainlanguage = czech,
	otherlanguages = {slovak,english},
	title-czech = {Detailní vyhodnocení přístupů k úloze Structure-from-Motion},
	title-english = {A Detailed Evaluation of State-of-the-Art Structure-from-Motion Approaches},
	% subtitle-czech = {Cesta do tajů kdovíčeho},
	% subtitle-english = {Journey to the who-knows-what wondeland},
	doctype = M,
	faculty = F3,
	% department-czech = {Katedra matematiky},
	department-english = {Department of Cybernetics},
	author = {Valentin Lhermitte},
	supervisor = {Torsten Sattler, Dr. rer. nat.},
	% supervisor-address = {Ústav X, \\ Uliční 5, \\ Praha 99},
	% supervisor-specialist = {John Doe},
	fieldofstudy-english = {Open Informatics},
	subfieldofstudy-english = {Computer Vision and Image Proccessing},
	% fieldofstudy-czech = {Matematcké inženýrství},
	% subfieldofstudy-czech = {Matematické modelování},
	keywords-czech = {Structure-from-motion, Moderní SFM, Vyhodnocení polohy kamery, Syntéza nového pohledu, Vyhodnocovací protokol},
	keywords-english = {Structure-from-motion, Modern SFM, Camera pose evaluation, Novel view synthesis, Evaluation protocol},
	day = 23,
	month = 5,
	year = 2025,
	specification-file = {ctutest-zadani.pdf},
%	front-specification = true,
%	front-list-of-figures = false,
%	front-list-of-tables = false,
%	monochrome = true,
%	layout-short = true,
}

\ctuprocess

\addto\ctucaptionsczech{%
	\def\supervisorname{Vedoucí}%
	\def\subfieldofstudyname{Studijní program}%
}

\ctutemplateset{maketitle twocolumn default}{
	\begin{twocolumnfrontmatterpage}
		\ctutemplate{twocolumn.thanks}
		\ctutemplate{twocolumn.declaration}
		\ctutemplate{twocolumn.abstract.in.titlelanguage}
		\ctutemplate{twocolumn.abstract.in.secondlanguage}
		\ctutemplate{twocolumn.tableofcontents}
		\ctutemplate{twocolumn.listoffigures}
	\end{twocolumnfrontmatterpage}
}

% Theorem declarations, this is the reasonable default, anybody can do what they wish.
% If you prefer theorems in italics rather than slanted, use \theoremstyle{plainit}
\theoremstyle{plain}
\newtheorem{theorem}{Theorem}[chapter]
\newtheorem{corollary}[theorem]{Corollary}
\newtheorem{lemma}[theorem]{Lemma}
\newtheorem{proposition}[theorem]{Proposition}

\theoremstyle{definition}
\newtheorem{definition}[theorem]{Definition}
\newtheorem{example}[theorem]{Example}
\newtheorem{conjecture}[theorem]{Conjecture}

\theoremstyle{note}
\newtheorem*{remark*}{Remark}
\newtheorem{remark}[theorem]{Remark}

\setlength{\parskip}{2ex plus 0.2ex minus 0.2ex}

% Abstract in Czech
\begin{abstract-czech}
	Techniky SfM (Structure-from-Motion) jsou nedílnou součástí rekonstrukce 3D geometrie z 2D obrazových sekvencí.
	Nedávný pokrok v oblasti SfM vedl k vývoji několika nejmodernějších algoritmů, které využívají hluboké učení a další moderní techniky.
	Tato zpráva hodnotí čtyři současné algoritmy SfM \textbf{GLOMAP, VGGSfM, FlowMap} a \textbf{AceZero} na rozmanitých referenčních souborech dat.
	Hodnocení se zaměřuje na kritické ukazatele, včetně přesnosti odhadu polohy kamery, kvality syntézy nových pohledů a také časové efektivity a spotřeby paměti.
	Poznatky z této studie mají za cíl nasměrovat budoucí vývoj a použití SfM v počítačovém vidění a příbuzných oborech.
\end{abstract-czech}

% Abstract in English
\begin{abstract-english}
	Structure-from-Motion (SfM) techniques are integral in reconstructing 3D geometry from 2D image sequences.
	Recent advancements in SfM have led to the development of several state-of-the-art algorithms that leverage deep learning and other modern techniques.
	This report evaluates four contemporary SfM algorithms \textbf{GLOMAP, VGGSfM, FlowMap} and \textbf{AceZero} on a diverse set of benchmark datasets.
	The evaluation focuses on critical metrics, including camera pose estimation accuracy, the quality of novel view synthesis as well as time efficiency and memory consumption.
	The insights from this study aim to guide future development and application of SfM in computer vision and related fields.
\end{abstract-english}

% Acknowledgements / Podekovani
\begin{thanks}
	I would like to thank Torsten Sattler, Dr. rer. nat., for his guidance and supervision throughout this work.

	I would also like to acknowledge the support of the OP VVV funded project \text{CZ.02.1.01/0.0/0.0/16\_019/0000765} \text{``Research Center for Informatics''}
	which provided the necessary resources and infrastructure for this research.
\end{thanks}

% Declaration / Prohlaseni
\begin{declaration}
	I, Valentin Lhermitte, hereby declare that this report is a true and original work completed by myself, 
	under the guidance and supervision of Torsten Sattler, Dr. rer. nat.
	The work presented in this report has not been submitted elsewhere for assessment
	and has not been previously used to obtain any academic qualification or degree. All
	references and sources used in this report have been duly acknowledged and cited in the bibliography.
	I affirm that the data, facts, and opinions presented in this report are true and accurate
	to the best of my knowledge and belief. I take full responsibility for the content of this
	report and any errors or omissions therein.

In Prague, \ctufield{day}.~\monthinlanguage{title}~\ctufield{year}
\end{declaration}

% Only for testing purposes
\listfiles
\usepackage[pagewise]{lineno}
\usepackage{lipsum,blindtext}
\usepackage{mathrsfs} % provides \mathscr used in the ridiculous examples
\usepackage{amsmath}
\usepackage{graphicx}
\usepackage{subcaption} % Required for subfigure captions
\usepackage{float} % Required for H float positioning
\usepackage{array} % Required for bold table lines
\usepackage[dvipsnames]{xcolor}
\usepackage{colortbl}
% \usepackage[backend=biber,style=numeric]{biblatex}
\usepackage{hyperref}
\usepackage{booktabs}     
\usepackage{multirow}
\usepackage{longtable}
\usepackage{siunitx}         % aligns numbers on the decimal point
\usepackage{adjustbox}       % quick resize or clip
\usepackage{rotating} 	  	 % for sidewaystable
\usepackage{siunitx}
\usepackage{changepage}

\sisetup{table-format=1.2, detect-all, round-mode=places}

\hypersetup{
	colorlinks=true,
	linkcolor=red,
	citecolor=blue,
	urlcolor=blue,
}

\newcommand{\best}[1]{\textbf{#1}}
\newcommand{\sbest}[1]{\underline{#1}}

% \addbibresource{src/references.bib}

\begin{document}

\maketitle

\section{Introduction}\label{sec:introduction}

Structure-from-Motion (SfM) is a vital technique in computer vision, enabling the reconstruction of 3D geometry from 2D image sequences. 
Its applications span diverse domains, including robotics, augmented reality, and photogrammetry. 
With a growing number of SfM algorithms, systematic evaluation becomes essential to assess their performance and applicability under varying scenarios.

This report builds upon a structured evaluation protocol to compare four contemporary SfM approaches: Glomap \cite{pan2024glomap}, VGGsfm \cite{wang2023vggsfm}, Flowmap \cite{smith24flowmap}, and Ace0 \cite{brachmann2024acezero}. 
Leveraging datasets such as ETH3D \cite{Schops_2019_CVPR}, Tanks and Temples \cite{Knapitsch2017}, and MipNeft360 \cite{barron2022mipnerf360}, 
this study benchmarks these algorithms using metrics such as camera pose error, 3D triangulation accuracy, novel view synthesis quality \cite{DBLP:journals/corr/WaechterBFMKG16} as well as time and memory efficiency.
These benchmarks aim to capture a holistic view of algorithm performance, considering both geometric accuracy and scene reconstruction fidelity.

Central to this evaluation is the computation of relative rotation error (RRE) and relative translation error (RTE) for assessing camera pose estimation. 
These metrics provide insights into the precision of relative camera transformations derived from SfM models. 
The evaluation protocol will be extended to incorporate further metrics, such as triangulation completeness and rendering accuracy for novel views.

This study aspires to provide actionable insights into the capabilities of modern SfM algorithms and guide their application across various datasets and domains. 
By dissecting their strengths and limitations, it aims to foster improvements in the design and deployment of SfM systems.
\chapter{Background and Related Work}\label{chap:related_work}

In this chapter, we provide a comprehensive overview of the foundational concepts and recent advancements relevant to Structure-from-Motion (SfM).
We begin by describing classical SfM techniques, highlighting traditional feature extraction, matching methods, and optimization strategies. 
Subsequently, we review emerging learning-based approaches, focusing on methods that leverage neural networks to enhance reconstruction accuracy, efficiency, and robustness. 
The chapter further delves into detailed discussions of specific state-of-the-art methods, including GLOMAP, VGGSfM, FlowMap, and AceZero, 
clarifying their claimed advantages and limitations compared to classical SfM pipelines like COLMAP.

\section{Classical SfM}
SfM approaches can be broadly classified into two categories: incremental and global.

\subsection{Incremental SfM}
Incremental Structure-from-Motion (SfM) follows a sequential approach, where the 3D reconstruction is built progressively by adding one image at a time. 
This method begins with an initial pair of images and expands the reconstruction iteratively by registering new images and refining the existing structure. 
The core steps of incremental SfM include feature extraction and matching, relative camera pose estimation, and bundle adjustment.

\paragraph{Feature Extraction and Matching}
The first step in incremental SfM involves detecting and describing key points in each image. 
Traditional methods rely on hand-crafted feature descriptors like SIFT \cite{Lowe2004DistinctiveIF} or ORB \cite{rublee2011orb}, 
which are designed to extract distinctive image features that are robust to changes in scale, rotation, and lighting conditions.

Once features are detected, they are compared across overlaping images to establish correspondences.
This is typically done using nearest-neighbor matching, where the descriptor of a feature in one image is compared to all features in another image to find the best match.
This process can be computationally expensive, especially for large datasets, and is often optimized using techniques like FLANN \cite{muja2009fast} or approximate nearest neighbor search.

\begin{figure}[h]
    \centering
    \includegraphics[width=0.8\textwidth]{figures/related_work/matches.jpg}
    \caption[Feature matching example]{Example of feature matching between two images. The lines connect matched features across the two images.}
    \label{fig:incremental_sfm}
\end{figure}


\paragraph{Relative Camera Pose Estimation}
Given the matched feature correspondences between image pairs, the next step is to estimate the motion of the camera between the two views.
This is typically done by computing the essential or fundamental matrix, which encapsulates the relative rotation and translation between the two camera poses.

This is achieved using epipolar geometry constraints, often solved via the five-point or eight-point algorithms within a RANSAC \cite{fischler1981random} framework to remove matches that are outliers.
The estimated relative camera pose allows for triangulation, reconstructing initial 3D points.

\paragraph{Bundle Adjustment}
As new images are added, the estimated structure and camera parameters can suffers from drift and inaccuracies.
Bundle adjustment is a key step in incremental SfM that refines the camera poses and 3D structure by minimizing the reprojection error.
Non-linear least squares solvers such as Levenberg-Marquardt or Ceres solver are commonly used to achieve this refinement. 
Incremental SfM benefits from high accuracy but suffers from scalability issues as the number of images increases.

\subsection{Global SfM}
In contrast to incremental approaches, global SfM processes all images simultaneously to compute a globally consistent 3D structure.
Instead of reapeting the costly bundle adjustment, global SfM methods estimate camera geometry for all input images at once.

A \emph{view graph} is constructed with all image pairs and their estimated relative poses.

The graph is then used to estimate camera intrinsics, as well as construction of a relative camera pose graph.
The relative camera pose graph is then used to perform \emph{rotation averaging} as well as \emph{translation averaging} \cite{Chatterjee2013, theia-manual, moulon2016openmvg}
Finally, with both global rotations and translations estimated, a global bundle adjustment is performed to refine the camera parameters and 3D point positions, ensuring a consistent and accurate reconstruction of the scene.
This approach allow for more scalable reconstructions.

\paragraph{However, three limitations remains:}
\begin{enumerate}
    \item \textbf{Colinear motion}: When the motion of the camera is colinear, it leads to degenerate reconstruction problem since the essential martrix only tells the direction of relative translation.
    This represent a challenge as such motion is frequently encountered in real-world scenarios (e.g., self-driving car or drone flying in a straight line).
    \item \textbf{Translation direction challenges}: The process of translation averaging is particularly sensitive to noise and outliers. 
    To accurately estimate the global camera position, triplet of relative translation directions are required. However, when these triplets form a skewed triangle, the translation averaging process becomes instable \cite{manam2023sensitivity}.
    \item \textbf{Camera intrinsics}: To decompose the relative two-view geometry into rotation and translation, the camera intrinsics must be known and accurate.
    If the intrinsics matrix $\mathbf{K}$ is not accuratly estimated, it can lead to large errors in the translation direction.
\end{enumerate}

\section{COLMAP}\label{sec:colmap}
COLMAP \cite{schoenberger2016sfm} is an open-source, modular SfM and Multi-View Stereo (MVS) system introduced by \textit{Schönberger and Frahm} in 2016. 
It implements a complete incremental reconstruction pipeline while offering dense reconstruction via PatchMatch stereo. COLMAP is designed for extensibility and high performance, and has become a widely adopted baseline in both academic and industrial 3D reconstruction tasks.

\paragraph{Feature Extraction and Matching}
COLMAP employs SIFT \cite{Lowe2004DistinctiveIF} feature detection and description for robust feature extraction. 
Since COLMAP is easily modular, RootSIFT \cite{arrandjelovic2012three} or more recent learning-based method such as SuperPoint \cite{detone18superpoint}, can be optionally used for improved matching robustness.
By default, exhaustive matching is used for small image sets, while for larger collections an image retrieval stage using a vocabulary-tree accelerates the selection of candidate image pairs. 
All putative matches are filtered via two-view geometric verification using fundamental or essential matrices estimated with RANSAC \cite{fischler1981random}.

\paragraph{Incremental Reconstruction Pipeline}
The incremental SfM pipeline proceeds as follows:
\begin{enumerate}
  \item \textbf{Initialization.} A well-conditioned image pair is chosen based on match count and baseline, and an initial two-view reconstruction is computed.
  \item \textbf{Image Registration.} Each subsequent image is registered by solving the Perspective-n-Point problem with RANSAC to estimate its pose relative to the growing 3D model.
  \item \textbf{Triangulation.} New 3D points are created by triangulating feature tracks that span at least two registered views.
  \item \textbf{Local Bundle Adjustment.} After each registration, a local bundle adjustment refines the poses of the newly added image and its neighbors to minimize reprojection error.
  \item \textbf{Global Bundle Adjustment.} Periodically, a full BA over all cameras and points is run to eliminate drift and optimize the entire reconstruction resulting in an amortized linear run-time.
\end{enumerate}

% \subsection{Dense Reconstruction}
% After sparse SfM, COLMAP can perform dense MVS via a GPU-accelerated PatchMatch stereo algorithm. For each reference image, a depth and normal map is estimated by propagating and refining plane hypotheses, followed by fusion into a global dense point cloud or mesh. This approach combines high accuracy with real-time performance for hundreds of views.

% \paragraph{Implementation Details}
% \begin{itemize}
%   \item \textbf{Solver.} All optimization (BA, stereo) leverages the Ceres Solver for efficient non-linear least squares.
%   \item \textbf{Data Structures.} COLMAP uses a database backend to store features, matches, and camera parameters, enabling on-disk handling of large datasets.
%   \item \textbf{Extensibility.} Each pipeline stage is exposed as a command-line tool or library API, allowing users to swap in custom feature detectors, matching strategies, or outlier filters.
% \end{itemize}

Overall, COLMAP provides a robust, end-to-end reconstruction framework that balances accuracy, scalability, and ease of integration, making it the de facto open-source choice for SfM and 3D reconstruction.

\section{GLOMAP}\label{sec:glomap}
GLOMAP \cite{pan2024glomap} presented by \textit{L. Pan et al.} is a recent global Structure-from-Motion method that aims to overcome the challenges of global SFM. It can deal with unknown camera intrinsics as well as colinear motion.

\paragraph{Feature Extraction and Matching}
The pipeline rely of COLMAP features extraction and matching, which use SIFT \cite{Lowe2004DistinctiveIF} for features extraction and has several option for matching: 
\begin{itemize}
    \item \textbf{Exhaustive}: All images are matched against each other, which is computationally expensive but provides the most accurate matches.
    \item \textbf{Sequential}: Matches are computed in a sequential manner, which is more efficient than exhaustive matching but requires sequential images.
    \item \textbf{Vocabulary Tree}: A Bag-of-Words images retrival approach \cite{schoenberger2016vote} that allows to find overlaping images in a large dataset and match them.
\end{itemize}
\textbf{Note:} The use of library such as Hloc \cite{sarlin2019coarse} can serve as a drop-in replacement for the feature extraction and matching step.

\paragraph{Camera Pose Estimation and Triangulation}
Since GLOMAP is a global SfM method, it seeks to estimate all the camera rotation and translation parameters simultaneously.
The global camera pose estimation is performed usually performed by a combination of rotation averaging and translation averaging. 
However, due to noise and outliers as well as scale ambiguity, translation averaging is particularly challenging.

GLOMAP introduces a novel approach and perform rotation averaging and a steps called \emph{Global positioning} instead of translation averaging. \\
Global positioning is the key step for robustness and perform directly a joint camera and point position estimation. 
Employing the BATA loss \cite{zhuang2019baselinedesensitizingtranslationaveraging} with a bounded reprojection error in the range $[0, 1]$. 
This allows the optimization process to converge quickly and prevent outliers from impacting the results.

By avoiding the translation averaging step, GLOMAP is able to handle colinear motion and better handle outliers.

Finally, global bundle adjustment is performed to refine the camera parameters and 3D point positions, ensuring a consistent and accurate reconstruction of the scene.

The authors claim superior performance and accuracy compared to traditional global SfM methods, and on par with COLMAP, while being significantly faster.

A limitation of GLOMAP remains. If the scene has a rotational symmetry, the rotation averaging step collapses, making it a degenerate case for GLOMAP.

\section{VGGSfM}\label{sec:vggsfm}
Visual Geometry Grounded Deep Structure From Motion (VGGSfM) \cite{wang2023vggsfm} presented by \textit{Wang et al.} introduces a novel, fully differentiable pipeline for Structure-from-Motion (SfM) that harnesses the potential of deep learning. The authors propose a unified framework that integrates the entire SfM pipeline into a function $f_{\theta}$, parameterized by $\theta$, which maps a set of images $\mathcal{I}$ to camera parameters $\mathcal{P}$ and a 3D point cloud $\mathcal{X}$. Here, $\theta$ is a learnable parameter, optimized by minimizing a loss function $\mathcal{L}$.

In its original formulation, $f_\theta$ can be decomposed into four stages: (1) Point Tracking, (2) Camera Estimator, (3) Triangulator, and (4) Bundle Adjustment.

\begin{figure}[h]
    \centering
    \includegraphics[width=1\textwidth]{figures/related_work/vggsfm_pipeline.png}
    \caption{VGGSfM pipeline overview \cite{wang2023vggsfm}}.
    \label{fig:vggsfm}
\end{figure}

\paragraph{Point Tracking.}
Unlike traditional SfM methods that compute pairwise correspondences and chain them into multi-image tracks, VGGSfM employs a deep feed-forward tracker to predict pixel-accurate multi-image correspondences directly from the input images. 
It typically selects a set of query points in one image (e.g.\ using SuperPoint \cite{detone18superpoint}, SIFT \cite{Lowe2004DistinctiveIF}, ALIKED \cite{Zhao2023ALIKED}, or a combination) and extracts their feature descriptors. These descriptors are matched across all other images using a multi-scale cost volume, encoding their similarity. 
A Transformer attends to all input frames jointly, predicting the 2D location of each query point in every view along with its visibility and confidence. 
In contrast to pairwise matching that relies on incremental chaining, VGGSfM's direct multi-image tracking reduces both complexity and drift errors.

\paragraph{Camera Estimator and Triangulator.}
The initial camera estimator and triangulator stages use deep Transformers to generate an initial estimate of the camera parameters $\mathcal{P}$ and the 3D points $\mathcal{X}$, as opposed to relying on rotation/translation averaging or incremental registration. 
In the original paper's pipeline, the network could leverage track features as well as image features to produce camera poses. All cameras and points are thus registered collectively, ensuring differentiability and bypassing many incremental SfM complexities.

\paragraph{Bundle Adjustment.}
Finally, the pipeline employs a differentiable second-order Levenberg-Marquardt optimizer from the Theseus library \cite{pineda2022theseus} to refine both the camera parameters and 3D points by minimizing reprojection error. 
Because bundle adjustment (BA) is fully differentiable, the entire SfM pipeline can be trained end-to-end, unlike in traditional approaches that use non-differentiable solvers.

\paragraph{Comparison with Global SfM}
VGGSfM is related to global SfM approaches but differs in three crucial aspects:
\begin{itemize}
    \item It learns to track points directly from the images, rather than performing pairwise matching.
    \item It uses a deep network to regress camera poses in a single pass, rather than rotation/translation averaging.
    \item It employs differentiable bundle adjustment to refine parameters in an end-to-end manner.
\end{itemize}

\subsection*{VGGSfM v1.1 Updates}
Starting from v1.1, the VGGSfM implementation separates the camera predictor from the point-tracking stage. 
Concretely, the camera predictor now relies \emph{only} on image features rather than requiring track features. 
This design choice arose in response to users feedback:
\begin{itemize}
    \item \textbf{Faster Pose Estimates.} Some scenarios benefit from real-time or near-real-time camera pose estimation without waiting for the (comparatively slower) track prediction.
    \item \textbf{Flexibility.} Users can opt to generate approximate camera poses quickly and refine them later (if desired) with track-based bundle adjustment.
\end{itemize}
This change does slightly reduce the initial pose accuracy—because track features are no longer used for the first pose guess—but in practice, errors remain small enough (e.g.\ $\approx 3^\circ$ mean rotational error and $6\%$ translational error) \cite{VGGSFM_GitHubIssue}. 
These errors can be effectively corrected via the subsequent BA stage if a high-accuracy reconstruction is needed.


While VGGSfM demonstrates robust performance on several benchmarks such as Co3D \cite{reizenstein21co3d}, IMC \cite{Jin_2020}, and ETH3D \cite{schoeps2017cvpr}, the authors acknowledge it does not (yet) handle large-scale datasets (thousands of frames) as readily as classical SfM pipelines. Ongoing work (VGGSfM v2) aims to address these scalability challenges while maintaining the end-to-end differentiability and strong performance.

\section{FlowMap}\label{sec:flowmap}
FlowMap \cite{smith24flowmap} presented by \textit{C. Smith et al.} is a recently introduced Structure-from-Motion method leveraging an end-to-end differentiable approach for estimating camera poses, intrinsics, and dense per-frame depth maps from video sequences. 
Unlike classical SfM methods such as COLMAP, FlowMap does not rely on explicit 3D point triangulation from sparse correspondences. 
Instead, it formulates SfM as a gradient-descent optimization problem, supervised purely by off-the-shelf optical flow and sparse point tracking correspondences.

\paragraph{Differentiable Optimization}
FlowMap minimizes a least-squares objective by comparing the optical flow generated from estimated depth, camera intrinsics, and poses against pre-computed optical flow and point tracks. 
Crucially, depth is parameterized using a neural network that maps input RGB frames to dense per-pixel depth maps, enabling consistent depth predictions across similar image patches. 
This approach helps FlowMap leverage geometric information robustly, even when patches are poorly constrained due to minimal motion or noise in correspondences.

Camera poses in FlowMap are computed analytically as the solution to an orthogonal Procrustes problem, aligning pairs of consecutive depth maps transformed by optical flow correspondences. 
This pose solver is differentiable, allowing gradient information to flow back into the depth network and correspondence weighting module.

\paragraph{Intrinsic Parameter Estimation}
FlowMap introduces a differentiable focal length estimator, softly selecting camera intrinsics from a set of candidate focal lengths based on optical flow consistency. 
After an initial soft selection stage, it switches to direct regression for focal length, benefiting from improved initialization and robust optimization.

\paragraph{Point Tracks for Robustness}
In addition to dense optical flow, sparse point tracks spanning longer sequences are employed. 
These tracks minimize drift over extended trajectories by providing long-term geometric consistency.

\begin{figure}[h]
    \centering
    \includegraphics[width=1\textwidth]{figures/related_work/flowmap_pipeline.png}
    \caption{FlowMap pipeline overview \cite{smith24flowmap}}.
    \label{fig:flowmap}
\end{figure}

\paragraph{Correspondence Loss}
The Correspondence Loss in FlowMap is a Camera-Induced Flow Loss, which use known correspondences to compute the reprojection error. 
To compute the loss, a points $\mathbf{u_i}$ is unprojected into the 3D space using the estimated depth map $\mathbf{D_i}$ and camera intrinsics $\mathbf{K_i}$, yielding a 3D point $\mathbf{X_i}$.
This point $\mathbf{X_i}$ is transfomed via the estimated relative camera pose $\mathbf{P_{ij}}$ to the second camera frame $\mathbf{I_j}$, where it is projected back into the second image plane.
The reprojection error is then computed as the difference between the projected point $\mathbf{u_j}$ and the known correspondence of $\mathbf{u_i}$ in the second image.


\paragraph{Comparison to Traditional SfM}
Unlike conventional methods, FlowMap produces dense depth estimates rather than sparse 3D points, a critical advantage for tasks like novel view synthesis. 
Empirical evaluations demonstrate that FlowMap achieves reconstruction quality comparable to COLMAP on standard benchmarks, such as Tanks \& Temples, Mip-NeRF 360, CO3D, and LLFF, especially when combined with downstream rendering approaches like Gaussian Splatting.

However, FlowMap does exhibit limitations. It relies heavily on accurate off-the-shelf correspondences, struggles with rotation-dominant trajectories, and can suffer from local minima such as hollow-face geometry inversions. 
Further, while it excels in scenarios involving continuous video data, it currently lacks mechanisms for processing unstructured image collections typical in classical SfM.

\section{AceZero}\label{sec:acezero}
AceZero \cite{brachmann2024acezero} presented by \textit{E. Brachmann et al.} introduces an innovative approach to Structure-from-Motion (SfM) by framing it as an incremental application and refinement of a learned visual relocalizer. 
This method diverges from conventional feature-matching SfM frameworks by employing scene coordinate regression, enabling the construction of implicit neural scene representations directly from unposed image collections without relying on local feature matching.

\paragraph{Scene Coordinate Regression}
At the core of AceZero is a scene coordinate regression method \cite{brachmann2023ace}, a learning-based technique that regresses the coordinates of 3D scene points directly from 2D image locations. 
This bypasses the need for traditional feature extraction and matching, significantly streamlining the process and potentially reducing error propagation associated with matching failures. Unlike other neural reconstruction approaches, AceZero does not require camera pose priors or sequentially ordered inputs, enhancing flexibility and applicability.

\paragraph{Incremental Learning of Visual Relocalizer}
AceZero iteratively refines a visual relocalizer model, initially starting from a single seed image with a known identity pose. 
This model incrementally registers additional views, continuously improving its scene understanding and pose estimation capabilities. 
Through iterative learning and refinement, AceZero can robustly scale to datasets comprising thousands of images.

\paragraph{Advantages and Limitations}
AceZero demonstrates competitive pose accuracy compared to classical feature-based SfM methods, as shown through evaluations on standard benchmarks, including novel view synthesis experiments. 
One significant advantage is its ability to implicitly represent scenes using neural networks, potentially capturing complex scene structures better than discrete point clouds.

However, as a learning-based approach, AceZero is reliant on sufficient training data to generalize well across various scenes and environments. 
The method's performance might degrade in scenarios with insufficient data coverage or highly repetitive visual textures. 
Furthermore, the scene coordinate regressor struggles with large-scale scenes, as well as big lighting changes such as day-night transitions.
AceZero also consider the camera intrinsics to be shared across all images, which may not be suitable for all datasets.

In summary, AceZero presents an exciting direction for neural-based SfM, promising improved robustness and implicit scene understanding capabilities, though it may face challenges in generalization and scalability compared to traditional methods.

\section{Gaussian Splatting for Novel View Synthesis}\label{sec:gaussian_splatting}
Gaussian Splatting \cite{kerbl20233dgaussiansplattingrealtime} is a recently introduced technique for rendering 3D scenes in real time by representing them with a set of continuous Gaussian functions.
Much like NeRF \cite{mildenhall2020nerf}, it can generate novel views from sparse images, but it departs from NeRF's dense sampling strategy and instead relies on a compact set of Gaussian “splats” to capture scene geometry and appearance.

\subsection{Scene Representation}
Instead of relying on dense voxel grids or explicit meshes, Gaussian Splatting represents a 3D scene as a collection of Gaussian functions. 
Each Gaussian splat is defined by its mean, which specifies the 3D center; its covariance, a $3 \times 3$ matrix that encodes the shape, scale, and orientation (often in an anisotropic manner); its associated RGB color; and a weight that modulates its contribution in the scene. 
This representation creates a continuous model of the scene where neighboring Gaussians naturally blend, leading to smooth visual transitions.

Formally, each Gaussian $g_i(\mathbf{x})$ is expressed as
\begin{equation}
    g_i(\mathbf{x}) = \sigma(\alpha_i) \exp\left(-\frac{1}{2} \bigl(\mathbf{x} - \boldsymbol{\mu}_i\bigr)^\top \Sigma_i^{-1} \bigl(\mathbf{x} - \boldsymbol{\mu}_i\bigr)\right),
\end{equation}

where $\boldsymbol{\mu}_i$ is the 3D center, $\Sigma_i$ is the covariance matrix, and $\sigma(\alpha_i)$ is a sigmoid function that modulates the Gaussian's opacity based on its weight $\alpha_i$.

\subsection{Volumetric Rendering and Ray Marching}
\paragraph{Ray Casting and Sampling:}
To render a novel view, Gaussian Splatting conceptually follows a ray-casting procedure akin to NeRF.
A ray is cast from each pixel, intersecting the cloud of Gaussians in 3D.
However, instead of sampling many points along the ray, the algorithm computes the projected contribution of each Gaussian onto the ray.
In practice, these Gaussians project to ellipses in the image, often referred to as “splats.”

\paragraph{Volume Rendering Integration:}
Like traditional volumetric rendering, the final color of each pixel is obtained by integrating the contributions of the Gaussians encountered along that ray.
Closer, high-density Gaussians have a stronger influence on the pixel color, while distant or low-density Gaussians are more transparent.
This process can be viewed as a weighted alpha-compositing step:

% \begin{equation}
%     \text{Color}(\mathbf{r}) = \sum_{i} T_{i-1} \,\alpha_i \,g_i(\mathbf{r}),
% \end{equation}

\begin{equation}
    \text{Color}(\mathbf{r}) = \sum_{i} \prod_{j=1}^{i-1}(1 - \alpha_j) \,\alpha_i \,g_i(\mathbf{r}),
\end{equation}

where $\alpha_i$ is the opacity $i$-th Gaussian, and $\prod_{j=1}^{i-1}(1 - \alpha_j)$ is the accumulated transmittance up to Gaussian $i-1$.

Because each Gaussian has a continuous extent (defined by its covariance), the representation is smooth and captures fine scene details without requiring dense point sampling.

\subsection{Learning and Optimization Process}
\paragraph{Training Objective:}
During training, a photometric loss between rendered images and ground truth images drives the optimization of each Gaussian's parameters (mean, covariance, color, and weight).
Typically, a mean squared error or a combination of L1 and SSIM loss is used to measure the difference between the rendered and target images.
The optimization process adjusts the Gaussian parameters to minimize this loss, effectively learning the scene representation.

\paragraph{Differentiable Rendering:}
Crucially, the Gaussian Splatting formulation is fully differentiable.
By computing analytic derivatives of each Gaussian's projection and integration, gradients can flow back through the splatting process.
This means the positions, shapes, and colors of the Gaussians are refined in an end-to-end manner, allowing the model to converge toward an accurate and compact representation of the scene.

\subsection{Novel View Synthesis}
Once trained, the collection of Gaussians can be used to render the scene from any new viewpoint by projecting them onto the new camera plane and compositing their contributions.
Despite the absence of a dense voxel or mesh structure, Gaussian Splatting achieves high-quality visual fidelity while often being more computationally efficient than sampling thousands of points along each ray (as done in many NeRF-style methods).
It also naturally handles partially transparent or semi-opaque regions because each Gaussian's opacity contributes smoothly during splatting.

\subsection{Advantages and Limitations}
\paragraph{Advantages:}
\begin{itemize}
    \item \textbf{Efficiency:} Fewer samples per ray are needed compare to NeRFs because each Gaussian describes a continuous region in space.
    \item \textbf{Smooth Blending:} Overlapping Gaussians create soft transitions, reducing aliasing and flickering in rendered views.
    \item \textbf{Differentiable Splatting:} The rendering process is end-to-end trainable, enabling data-driven optimization.
    \item \textbf{Real-Time Potential:} With the right GPU-based implementations, Gaussian Splatting can achieve near-real-time rendering performance.
\end{itemize}

\paragraph{Limitations:}
\begin{itemize}
    \item \textbf{Sensitivity to Initialization:} Selecting initial Gaussian parameters can be challenging; poor initialization may lead to suboptimal solutions.
    \item \textbf{Complex Scenes:} Highly detailed or large-scale scenes may require many Gaussians, increasing both memory usage and render time.
    \item \textbf{Dependence on Pose Accuracy:} As with NeRF, reliable camera poses (or a reliable pose optimization stage) are essential for achieving high-quality results. That is why accurate SfM is crucial.
\end{itemize}

\subsection{Relation to Structure-from-Motion (SfM) Techniques}
Like NeRF, Gaussian Splatting requires knowledge of camera intrinsics and extrinsics to project Gaussians correctly onto the image plane.
SfM pipelines such as GLOMAP, VGGSfM, FlowMap, and AceZero can provide these camera poses from unstructured image sets or video.
Once the poses are available, Gaussian Splatting can be trained or refined using the resulting images.
This synergy allows for:
\begin{itemize}
    \item \textbf{Accurate Geometry Initialization:} SfM's 3D points or camera parameters help position Gaussians in the correct locations.
    \item \textbf{Improved Novel View Synthesis:} The continuous Gaussian representation can fill in details and handle transparent or semi-transparent regions better than sparse point-based methods alone.
\end{itemize}

\begin{figure}[H]
    \centering
    \includegraphics[width=\textwidth]{figures/related_work/gs_splat_pipeline.png}
    \caption{Gaussian Splatting pipeline overview \cite{kerbl20233dgaussiansplattingrealtime}.}
    \label{fig:gaussian_splatting}
\end{figure}

Overall, Gaussian Splatting offers an efficient and differentiable avenue for novel view synthesis, complementing existing SfM reconstructions by generating high-fidelity renders at interactive speeds.
\chapter{Methodology}\label{chap:methodology}

This chapter describes the evaluation protocol used to assess the performance of the four SfM pipelines: AceZero, GLOMAP, VGGSfm, and FlowMap.
We first present the proposed evaluation protocol. Then, we describe the datasets used for evaluation, including their characteristics and the rationale behind their selection.

\section{Proposed evaluation protocol}\label{sec:proposed-evaluation-protocol}

In this section, we present a comprehensive evaluation protocol for assessing the performance of Structure from Motion (SfM) algorithms.
The proposed evaluation protocol consists of four main components: Camera Pose Evaluation, Novel View Synthesis Evaluation, Time performance, and GPU memory usage.
These components are designed to evaluate the accuracy of camera pose estimation and the quality of the reconstructed scene geometry, respectively.

\subsection{Camera Pose Evaluation}
To quantitatively assess camera pose accuracy, we compute relative rotations and translations between all possible pairs of estimated camera poses and compare these to the corresponding ground-truth poses.
Additionally, we compare the absolute errors in camera poses between the estimated and ground-truth poses.

\subsubsection{Relative Pose Error}
Given two cameras \( i \) and \( j \), let the relative rotation and translation from camera \( i \) to camera \( j \) be defined as:
\begin{equation}
    R_{ij} = R_j R_i^T
\end{equation}

\begin{equation}
    \vec{t}_{ij} = \vec{t}_i - R_{ij} \vec{t}_j
\end{equation}

where \( R_i \) and \( R_j \) are the rotation matrices of cameras \( i \) and \( j \) respectively, and \( \vec{t}_i \) and \( \vec{t}_j \) are their respective translation vectors.

\paragraph{Relative Translation Error (RTE)}
The Relative Translation Error (RTE) measures the angular difference between the estimated and ground-truth camera relative translations and is computed as:

\begin{equation}
    \text{RTE}_{ij} = \cos^{-1}\left(\frac{\vec{t_{ij}}^{\text{gt}} \cdot \vec{t_{ij}}^{\text{est}}}{\lVert \vec{t_{ij}}^{\text{gt}} \rVert \lVert \vec{t_{ij}}^{\text{est}} \rVert}\right)
\end{equation}

\paragraph{Relative Rotation Error (RRE)}
The Relative Rotation Error (RRE) quantifies the angular difference between estimated and ground-truth relative rotations. It is defined as:

\begin{equation}
    \text{RRE}_{ij} = \cos^{-1}\left(\frac{\text{trace}(R_{ij}^{\text{gt}} (R_{ij}^{\text{est}})^T) - 1}{2}\right)
\end{equation}


if a pose is missing, we apply a high penalty of 180 degrees to both the RTE and RRE.


\paragraph{Area Under the Curve (AUC)}
To summarize the performance of the SfM algorithms, we report the Area Under the Curve (AUC) for the cumulative distribution of angular errors at different thresholds. 
This provides a comprehensive assessment of the algorithm's accuracy across varying levels of tolerance.

\subsubsection{Absolute Pose Error}
The Absolute Pose Error measures the absolute difference between the estimated and ground-truth camera poses.

In order to compute it, we first need to align the models. Indeed, the estimated reconstruction might be in a different coordinate system than the ground-truth reconstruction.

We align the estimated reconstruction with the ground-truth reconstruction using COLMAP \textit{model aligner}.

After alignment, the error in pose can be decomposed into two parts: the \textbf{rotation error} and the \textbf{camera position (translation) error}.

The camera position in the world coordinate system is given by:
\begin{equation}
    \vec{C} = -R^\top \cdot \vec{t}
\end{equation}
where the rotation matrix $R$ is the world orientation in the camera coordinate frame and the translation vector $\vec{t}$ is the world origin in the camera coordinate frame.

To compute the camera position error, we compare the estimated camera center $\vec{C}_{\text{est}}$ to the ground-truth camera center $\vec{C}_{\text{gt}}$ using the Euclidean distance:
\begin{equation}
    \vec{C}_{\text{error}} = \left\| \vec{C}_{\text{est}} - \vec{C}_{\text{gt}} \right\|_2
\end{equation}

The rotation error is computed using the relative rotation matrix:
\begin{equation}
    R_{\text{error}} = R_{\text{est}} R_{\text{gt}}^\top
\end{equation}

The angle of rotation $\alpha$ between the estimated and ground-truth rotations is then calculated as:

\begin{equation}
    \alpha = \cos^{-1}\left( \frac{\text{trace}(R_{\text{error}}) - 1}{2} \right)
\end{equation}

This gives the angular difference in radians. The result can be converted to degrees if needed.


Again, if a pose is missing, we apply a high penalty of 180 degrees to the rotation error and a high penalty of 10 meters to the translation error.


\subsection{Novel View Synthesis Evaluation}

To evaluate the quality and accuracy of the reconstructed scene geometry, we employ Novel View Synthesis using Gaussian Splatting.
Specifically, the \texttt{GSplat} \cite{ye2024gsplatopensourcelibrarygaussian} framework is utilized to render synthetic images from novel camera viewpoints not included in the original set of images used for reconstruction. 
GSplat is an open-source library that implements Gaussian Splatting for real-time rendering of neural radiance fields (NeRFs) and is insipred by the original work on Gaussian Splatting \cite{kerbl20233dgaussiansplattingrealtime} but was imroved to be faster and more memory efficient.

3D Gaussian Splatting methods have recently demonstrated state-of-the-art fidelity and real-time performance, motivating their adoption in place of Neural Radiance Fields.

The quality of these synthetic views is quantitatively assessed using standard metrics, including Peak Signal-to-Noise Ratio (PSNR), Structural Similarity Index Measure (SSIM), and Learned Perceptual Image Patch Similarity (LPIPS). 

\begin{itemize}
    \item \textbf{PSNR}: Measures the ratio of peak signal power to noise power in the image, expressed in decibels (dB). Higher values indicate lower distortion.
    \item \textbf{SSIM}: Evaluates perceptual similarity by comparing luminance, contrast, and structure between images. A higher SSIM score implies a closer match to the reference image.
    \item \textbf{LPIPS}: Uses deep learning-based feature comparisons to assess perceptual similarity. Lower LPIPS scores indicate greater visual fidelity. AlexNet \cite{krizhevsky2012imagenet} is used as feature extractors for LPIPS.
\end{itemize}

Higher PSNR and SSIM scores, along with lower LPIPS scores, indicate superior reconstruction fidelity and scene representation accuracy.

Novel View Synthesis is a convenient way to evaluate the quality of the reconstructed scene geometry, as it doesn't require any additional data or ground-truth information.
Only the set of images used for reconstruction is needed, and the evaluation can be performed on any scene.

\subsubsection{Split Training and Evaluation Sets}\label{sec:split_sets}

We first sort the images according to their acquisition order and assign every
\emph{eighth} frame to the evaluation split, resulting in a deterministic $90\%$ training and $10\%$ evaluation split.
Formally, let $\mathcal{I}=\{I_1,\dots,I_N\}$ be the set of frames that were successfully SfM-registered by a given pipeline together with their camera poses
$\{\mathbf{T}_1,\dots,\mathbf{T}_N\}$. \\
\\
The training set $\mathcal{T}$ and evaluation set $\mathcal{E}$ are defined as:
\begin{equation}
  \mathcal{T} = \{I_k, \mathbf{T}_k\}_{k \in \mathcal{I}, k \mod 8 \neq 0}\,
\end{equation}
\begin{equation}
  \mathcal{E} = \{I_k, \mathbf{T}_k\}_{k \in \mathcal{I}, k \mod 8 = 0}\,
\end{equation}
The training set $\mathcal{T}$ is used to train the Gaussian Splatting model, while the evaluation set $\mathcal{E}$ is used to evaluate the performance of the SfM pipeline.

\paragraph{Handling missing registrations.}
In practice, the four tested SfM pipelines do not register exactly the same subset of images.  
A naïve stride-8 split would therefore evaluate each pipeline on a slightly different set of viewpoints, making the numerical comparison unfair.
Indeed, if a methods does not register a smaller amount of images compare to other SfM methods, this would results in a smaller subset $\mathcal{E}$ and thus the mean PSNR, SSIM and LPIPS would be the results of fewer images evaluated.

To guarantee that every algorithm is scored from identical viewpoints, we fill the gaps in $\mathcal{E}$: 
if an image $I_k$ is missing for a particular pipeline, we use the RGB content of $I_k$ together with the pose of the closest registered frame,
\begin{equation}
  \mathbf{T}_{\operatorname*{arg\,min}_{j\in\mathcal{I}} |j-k| }\,
\end{equation}

Applying this rule implies that some ground truth images will be compared with reconstructed images from a different pose that is the closest. 
Consequently, the resulting PSNR, SSIM, and LPIPS values will be affected by this heuristic.

However, we discovered that this approach strikes a good balance between fairness and evaluation consistency. 
It also serves as an effective way to penalize the absence of poses.

Since this substitution is applied symmetrically to all pipelines, the visual content is shared, and the evaluation remains fair.

\paragraph{No pose imputation during training.}
For the Gaussian-Splatting optimisation itself we deliberately \emph{do not} impute the missing poses: inaccurate or duplicated camera parameters act as label noise and noticeably hinder convergence. 
Consequently, each pipeline is trained on its own (possibly reduced) set $\mathcal{T}$, but all are evaluated on the common set $\mathcal{E}$. 
Empirically, we found that this strategy offers the best trade-off between training stability and the comparability of the final metrics.

% \subsection{3D Triangulation Evaluation}

% To evaluate the quality of the reconstructed 3D points, we follow \cite{Knapitsch2017} and compute the distance between the estimated dense points cloud and the ground-truth points cloud provided by the datasets.
% This is done by aligning the estimated points cloud with the ground-truth points cloud. We again use COLMAP \textit{model aligner} to compute an inital aligment and then perform Iterative closest point (ICP) \cite{Besl1992} to refine the alignment.

% To compute the distance between the points clouds, we estimate the per-point distances in both directions: from the reconstructed point cloud to the ground-truth point cloud, and vice versa. 
% This bidirectional comparison captures both completeness and accuracy. For each point in one cloud, we compute the Euclidean distance to its nearest neighbor in the other cloud by building a K-d search tree.

% We report the precision and recall of the reconstructed points cloud with respect to the ground-truth points cloud under a threshold distance of $0.1$m.

% \begin{itemize}
%     \item \textbf{Precision}: This measures the accuracy of the reconstructed points. Specifically, it is the fraction of reconstructed points that lie within $0.1$m of any point in the ground-truth point cloud. 
%     A high precision indicates that most of the reconstructed points are close to the true surface.
%     \begin{equation}
%         \text{Precision} = \frac{\sum_{i=1}^{N} \mathbb{I}(d_i < 0.1)}{N}
%     \end{equation}
%     where $d_i$ is the distance from the $i$-th point in the reconstructed point cloud to its nearest neighbor in the ground-truth point cloud, and $N$ is the total number of reconstructed points.

%     \item \textbf{Recall}: This measures the completeness of the reconstruction. It is the fraction of ground-truth points that are within $0.1$m of any point in the reconstructed point cloud.
%     A high recall indicates that most of the actual surface has been captured by the reconstruction.
%     \begin{equation}
%         \text{Recall} = \frac{\sum_{j=1}^{M} \mathbb{I}(d_j < 0.1)}{M}
%     \end{equation}
%     where $d_j$ is the distance from the $j$-th point in the ground-truth point cloud to its nearest neighbor in the reconstructed point cloud, and $M$ is the total number of ground-truth points.
% \end{itemize}


\subsection{Time and Memory Performance}
In addition to reconstruction quality, we report the computational performance of the evaluated SfM algorithms in terms of both time and memory efficiency. 
These aspects are particularly important when considering deployment in real-world applications, especially those requiring real-time processing or operating on large-scale datasets.

Time performance is measured as the total processing time needed to reconstruct a scene from the input images, which includes camera pose estimation and sparse point cloud generation. 
This metric reflects the overall computational efficiency of the algorithm.

Memory performance is assessed by recording the peak GPU memory usage during reconstruction. Memory efficiency is a critical factor for scalability, as not all users have access to high-end GPUs with large memory capacities. 
Such GPUs are not only expensive, but also increasingly difficult to obtain due to sustained high demand over the past few years in both academic and industrial settings. 
Efficient memory usage therefore enables broader accessibility and allows SfM algorithms to run on more modest hardware, facilitating adoption in resource-constrained environments.


\section{Dataset Selection}\label{sec:dataset-selection}

In this section, we provide an overview of the datasets used for evaluation and the rationale behind their selection.
The chosen datasets span a diverse range of scenarios, including both indoor and outdoor environments, varying lighting conditions, and different types of scene geometry and motion trajectories.
Some datasets feature long-range trajectories across large environments (e.g., urban or campus-scale), while others involve constrained camera motion around a central object or scene.

It is important to note that AceZero assumes consistent camera intrinsics across all images, which limits its applicability to datasets where intrinsics vary between frames.
Therefore, we excluded datasets such as IMC \cite{Jin2020}, which include multiple devices or cameras with different intrinsic parameters.

\subsection{ETH3D Stereo}
The ETH3D Stereo dataset \cite{schoeps2017cvpr} is a widely used benchmark for evaluating Structure-from-Motion (SfM) and stereo reconstruction algorithms. 
We selected 12 scenes from the high-resolution subset, which includes both indoor and outdoor environments. 
The camera captures images of either a large scene while retraining a relatively small number of images (e.g., 10 to 70 images).
All images within a scene are captured using the same DSLR camera and lens, ensuring consistent camera intrinsics throughout. 
The dataset provide ground-truth camera poses, but it does include high-quality 3D scan models of the scenes, which can be used for evaluating the accuracy of the reconstructed 3D points.

\subsection{LaMAR}
The LaMAR dataset \cite{sarlin2022lamar} is a large-scale benchmark designed for evaluating localization and mapping algorithms in augmented reality scenarios.
It includes a wide range of challenging environments with both indoor and outdoor scenes, complex geometry, and varied lighting conditions.
We use three large-scale scenes, each divided into multiple sessions. Each session contains a sequence of images captured with the same iOS device, ensuring consistent intrinsics.
While LaMAR also includes data from the Microsoft HoloLens, we only use the iOS device images to comply with the single-intrinsic constraint of AceZero. 

Similarity to ETH3D, the LaMAR dataset captures long-range trajectories, where the camera moves around a large scene, such as a building or a campus. 
However, it provide much more images, allowing to both highlights the performance of the SfM algorithm and their scalability.
The dataset provides high-quality ground-truth camera poses, making it suitable for evaluating camera pose estimation and robustness in real-world, dynamic environments.

\subsection{MiPNeRF360}
The MiPNeRF360 dataset \cite{barron2022mipnerf360} is designed for evaluating neural radiance field (NeRF) models in unbounded, 360-degree scenes.
It consists of several scenes where the camera orbits around a central object or region of interest, capturing a dense set of images from multiple viewpoints.
All scenes are captured under controlled lighting with consistent camera intrinsics, making it compatible with AceZero. 
This dataset is especially valuable for testing systems on closed-loop trajectories with complex object geometry and view-dependent effects.
Ground-truth camera poses are provided, enabling detailed evaluation of pose accuracy and consistency.

\subsection{Tanks and Temples}
The Tanks and Temples dataset \cite{Knapitsch2017} is a standard benchmark for evaluating multi-view stereo and 3D reconstruction techniques.
It consists of high-resolution image sequences of complex scenes such as statues, monuments, and architectural structures. 
Each sequence captures the camera rotating around a central subject, with sufficient overlap and varying viewpoints.
Scenes are recorded under natural outdoor lighting, introducing realistic challenges such as shadows and specularities.
All images within a scene share the same camera intrinsics, and accurate ground-truth 3D reconstructions are available, allowing both qualitative and quantitative assessments of SfM pipelines.
\chapter{Experiments}\label{chap:experiments}

In this section, we present the experimental setup and results of our evaluation of the different methods for 3D reconstruction and novel view synthesis. 
We compare the performance of VGGSfM, GLOMAP, AceZero, and FlowMap on selected datasets, including ETH3D, LaMAR, MipNeRF360, and Tanks and Temples.

\section{Experimental Setup}\label{sec:experimental-setup}
\subsubsection{Hardware}
All methods were evaluated on a HPC server, equiped with NVIDIA Tesla V100 GPUs (32 GB). Only a single V100 was used to ensure a level playing field as VGGSfM and AceZero simply don't offer multi-GPU support.
GLOMAP, which mirrors COLMAP's GPU-accelerated feature extractor and matcher, but its core mapping (bundle adjustment, graph construction, etc.) remains CPU-only in the current codebase. 
For that stage we provisioned 32 CPU threads on an Intel Xeon Scalable Gold 6150; in practice, however, CPU-core count has negligible impact on the other pipelines, since they execute end-to-end on the GPU.

\subsubsection{Software}
We made sure that all methods would produce the results in a COLMAP-compatible format (\textit{images.bin, cameras.bin, points3D.bin}) to make it easier to handle the data and to compare the results.
A Python-based pipeline was created and used pycolmap to handle the data throughout the process.

\subsubsection{Gaussian Splatting parameters}
The gaussian splatting was trained using 30 000 iterations and camera poses optimization enabled.

If a method use a camera model that handle lens distortion, we undistort the images using the OpenCV library \cite{opencv_library} before feeding them to the Gaussian Splatting pipeline.

\subsection{SfM methods parameters}
\paragraph{COLMAP}
It was run with a \textit{RADIAL} camera model and the \textit{Exhaustive Matcher} feature matcher unless specified otherwise.

The COLMAP version used for this experiment is 3.11.1.

\paragraph{VGGSfM}
VGGSfM was run with camera type set to \textit{SIMPLE RADIAL} and the default image keypoints and descriptors extractor \textit{ALIKED} \cite{Zhao2023ALIKED}.

We did however reduce the number of predicted tracks from 163.840 to 40.960, as well as the number of triangulated tracks from 819.200 to 204.800 
in order to reduce the memory usage since VGGSfM is indended to run on a a larger GPU with higher memory.

For this experiment, we use VGGSfM 2.0.

\paragraph{GLOMAP}
GLOMAP feature extractor was set to COLMAP's feature extractor \textit{SIFT} \cite{Lowe2004DistinctiveIF} and feature matcher to COLMAP's \textit{Exhaustive Matcher} unless specified otherwise.
We use the \textit{RADIAL} camera model.

For this experiment, we use GLOMAP version 1.0.0.

\paragraph{AceZero}\label{sec:acezero-parameters}
AceZero was run with the default parameters, which include the use of \textit{SIMPLE PINHOLE} camera model. 
This could represente a limitation for the method as it is not able to handle lens distortion.

AceZero provides a useful confidence score. Following the authors' recommendation, if a pose has a confidence score lower than 1000, we consider the pose as unreliable and set it to an unregistered image.

\paragraph{FlowMap}
To run FlowMap, we used the provided pretrained model which was trained on CO3D \cite{reizenstein21co3d}, Real Estate 10K, and KITTI \cite{geiger2012kitti} datasets.
This pre-training claims to achieve faster convergence and slightly improves the camera poses estimation.

While the pre-training has been done using GMFlow \cite{xu2022gmflow} for optical flow, we used the default \textit{RAFT} \cite{teed2020raft} optical flow model for the evaluation.

FlowMap also includes a low-memory mode, which reduces the resolution at which optical flow is computed. 
For scenes containing more than 180 images, we enable this mode by setting the corresponding parameter to \textit{True}. 
Through empirical evaluation on our GPU configuration, we found that 180 images serves as a practical threshold to prevent out-of-memory issues during processing.

\subsection{Datasets}
We evaluate the performance of different methods in estimating camera poses using the ETH3D and LaMAR datasets, both of which provide ground truth pose annotations.

\paragraph{LaMAR}
For the LaMAR dataset, we do not rely on the provided tools, as they are not well-suited for methods such as AceZero and FlowMap. Instead, we execute each Structure-from-Motion (SfM) pipeline manually on individual iOS sessions. 
Each session contains a large number of images and was captured in a sequential manner, which allows certain methods to take advantage of sequential processing.

In particular, we run the following methods configuration on the LaMAR dataset:

\begin{itemize}
    \item \textbf{COLMAP and GLOMAP:} These methods benefit from using the \textit{Sequential Matcher}, which significantly reduces computation time compared to the \textit{Exhaustive Matcher}.
    \item \textbf{VGGSfM:} Can operates in sequential mode using a sliding window approach, which lowers memory requirements and enables the reconstruction of longer sequences. 
                           But we found during our experiments that the sliding windows got stuck in an infinite loop while trying to find valid frames for the LaMAR dataset. Or the bundle adjustment would fail, resulting in a crash.
                           Thus, we do not leverage the sequential mode for this experiment.
    \item \textbf{FlowMap:} Naturally designed for sequential processing; no changes in configuration are necessary.
    \item \textbf{AceZero:} Lacks support for sequential processing. We therefore apply the same parameters as detailed in Section~\ref{sec:acezero-parameters}.
\end{itemize}

The novel view synthesis results are evaluated on the datasets ETH3D, MipNeRF360 and Tanks and Temples. 
As shown in section \ref{sec:experimental-results}, we use MipNeRF360 with an image resolution downscaled from the original $4949 \times 3286$ to $2473 \times 1643$

\section{Experimental Results}\label{sec:experimental-results}

\subsection{Camera Pose Estimation Results}\label{sec:camera-pose-estimation-results}

\subsubsection{ETH3D}\label{sec:eth3d-evaluation-results}
\input{src/poses/traj_results_eth3d.tex}

\subsubsection{LaMAR}\label{sec:lamar-evaluation-results}
\input{src/poses/traj_results_lamar.tex}

\subsection{Novel View Synthesis Results}\label{sec:gs-evaluation-results}
\input{src/nvs/nvs_results_eth3d.tex}
\input{src/nvs/nvs_results_mp360.tex}
\input{src/nvs/nvs_results_t2.tex}

\paragraph{Take-away.}
Overall, GLOMAP and COLMAP remains the top performers for the novel view synthesis task, 
AceZero and VGGSfM also produced honorable results on MipNeRF360 and Tanks and Temples, but they are not able to produce good results on ETH3D.
FlowMap remains the worst perfroming method and results shown by the authors do not match our results.

\subsection{Time And Memory Evaluation}\label{sec:time-and-memory-evaluation}
\input{tables/timings/eth3d_timings.tex}

\input{tables/timings/mipnerf360_timings.tex}

\input{tables/timings/tanksandtemples_timings.tex}
\input{tables/timings/tanksandtemples_reduced_timings.tex}

Tables [\ref{tab:time_performance_ETH3D}, \ref{tab:time_performance_MipNerf360}, \ref{tab:time_performance_TanksAndTemples}, \ref{tab:time_performance_TanksAndTemples reduced}]
clearly shows that GLOMAP is the fastest method by a significant margin.
COLMAP remains fast for small datasets compared to VGGSfM but quickly becomes out-paced for larger datasets.
AceZero is the slowest on average, but it is worth noting that as the number of images increases, the time performance of AceZero improves.
FlowMap running time is faster than AceZero on ETH3D but slower on the reduced version of Tanks and Temples.
However, given the results shown in section \ref{sec:camera-pose-estimation-results} and \ref{sec:gs-evaluation-results}, Flowmap's time performance is not worth the trade-off in accuracy.

\input{tables/timings/lamar_cab_timings.tex}
\input{tables/timings/lamar_hge_timings.tex}
\input{tables/timings/lamar_lin_timings.tex}

The LaMAR dataset tells the same story. GLOMAP demonstrates faster convergence than the other methods. 
The reason COLMAP shows very few results, is explained bellow in section \ref{sec:note}. However, \cite{pan2024glomap} and \cite{brachmann2024acezero} report that COLMAP becomes much slower when high number of images are involved. 
Which is to be expected from a incremental SfM pipeline.

For LIN \ref{tab:time_performance_LaMAR LIN}, GLOMAP seems more consistent with the results from [\ref{tab:time_performance_ETH3D}, \ref{tab:time_performance_MipNerf360}, \ref{tab:time_performance_TanksAndTemples}, \ref{tab:time_performance_TanksAndTemples reduced}].

AceZero runtime depends largly on the scenes, the number of iterations depends on the scene complexity.
Indeed, AceZero's paper \cite{brachmann2024acezero} states that the method is designed to be efficient for large datasets but the runtime complexity depends on two factors: the number of images and the spatial distribution of cameras.
In the worst case, the complexity is $O(n^2)$ where n is the number of images (e.g a camera trajectory without intersection or loops). 
The best case is $\Omega(n)$ when the cameras are well distributed in space.
The authors also claim much faster performance than COLMAP, on very large datasets, (e.g. 10k+ images with dense coverage of the scenes). 
But we did not have the opportunity to test this claim due to time constraints.

VGGSfM and FlowMap results are to be taken with caution as they only ran on very few sessions due to GPU memory constraints.

\paragraph{Take-away.}
GLOMAP is the fastest method for most datasets, and COLMAP remains a good option. 
\textit{L. Pan et al} \cite{pan2024glomap} reports that the method largely outperforms COLMAP on large datasets (a few thousands images) during the camera pose estimation and 3D mapping tasks.
VGGSfM's results are satisfactory for small to medium sized datasets.
FlowMap and AceZero remains the slowest

\subsubsection{Note}\label{sec:note}
Any scene not presented for COLMAP, are due to COLMAP unable to find a good initial image pair to start the reconstruction. 
In that case we report the time and memory usage when COLMAP manage to register at least a good amount of the images, 
otherwise it would be taking into account very small models (e.g. 10 images), which would report very fast runtime in the tables above and skew the results.
% ------------------------------------------------------------
% Things to include in discussion : 

% AceZero seems to perform well when the trajectory is a turns around a subject such as MipNeRF360. or purly outdoor/indoor such as LaMAR but no indoor/outdoor transition.
% AceZero time complexity depends on the number of images and the spatial distribution of cameras. Worst case is $O(n^2)$ where n is the number of images. best case is $\Omega(n)$ when the cameras are well distributed in space.

% FlowMap does not scale very well with the number of images. The video memory usage increases linearly with the number of images.
% FlowMap also assume to run on video sequences with small jumps between frames because it needs to compute optical flow.
% Indeed, the current implementation of FlowMap uses full batch processing. A possible improvement would be using mini-batches to help reduce the memory usage.
% Impossible to run LaMAR dataset on FlowMap as each sessions is a few hundreds of images.

% FlowMap current implementation uses full batch for computing optical flow. Perhaps using mini-batches could help reduce the memory usage and improve scalability.

% Unfortunately, reducing image size for vggsfm doesn't appear to have a real impact on the memory usage.

% COLMAP and GLOMAP could be improved by replacing the basic Sift based feature extraction with a more recent and efficient feature extractor such as SuperPoint \cite{detone2018superpoint}. The Hloc \cite{Hloc} library already provide a wrapper for SuperPoint and easy integration with COLMAP.

% ------------------------------------------------------------

\chapter{Discussion}\label{chap:discussion}
Across four heterogeneous benchmarks (ETH3D, LaMAR, MipNeRF360 and Tanks And Temples) a consistent ranking emerges:

\begin{itemize}
    \item \textbf{GLOMAP} delivers the best \emph{relative} pose accuracy on ETH3D (AUC$_{30}=93.4$) and got \SI{99.5}{\percent} accuracy, whereas COLMAP stops at \SI{75.6}{\percent}.
    \item \textbf{COLMAP} remains the most robust when absolute accuracy is critical: it attains the lowest median translation errors on LaMAR and remains faithful Gaussian-splat renders.
    \item \textbf{VGGSfM} achieves mid-tier accuracy but is memory hungry (\SI{>30}{GB} on ETH3D scenes) and fails on long LaMAR sequences due to sliding-window divergence.
    \item \textbf{AceZero} scales gracefully with sequence length but its worst-case time complexity is $O(n^{2})$ when camera baselines form a line; it also assumes a single intrinsic calibration and under-performs in wide environement with low images overlap (e.g. ETH3D, LaMAR).
    \item \textbf{FlowMap} struggles whenever optical-flow baselines exceed a few pixels. Even with ideal conditions (Tanks And Temples reduced), it fails to match the other methods accuracy.
\end{itemize}

\paragraph{GLOMAP}
Its joint \emph{global positioning} step side-steps translation averaging issues and explains the striking ETH3D gains.  
Nevertheless, a degenerate solutions produce metre-scale outliers (e.g.\ a \SI{105}{m} error in \texttt{courtyard}) inflating its mean translation error to \SI{9.1}{m}.
Because GLOMAP still relies on SIFT, scenes with motion blur or specular glass façades (common in LaMAR) occasionally break its feature graph; substituting SuperPoint \cite{detone18superpoint} through Hloc \cite{sarlin2019coarse} could reduce such failure modes.
The current implementation performs the mapping step on the CPU which could benefit from GPU acceleration.

\paragraph{COLMAP}
The incremental pipeline remains a safe baseline: it discards dubious keyframes aggressively, which keeps per-frame errors low but fragments long mobile sessions into multiple sub-models.
Replacing exhaustive matching with \emph{sequential} matching halves run-time on phone videos with a small accuracy drop, yet global pipelines still out-pace it on large-scale data.

\paragraph{VGGSfM}
The fully differentiable design is attractive for end-to-end learning, but its memory footprint scales with the amount of predicted tracks; even after reducing tracks $4\times$, MipNerf360 scenes need \SI{30}{GB} of VRAM.
On LaMAR, the tracker occasionally enters an infinite loop when no keyframe in the sliding window satisfies its visibility heuristics.
Enabling muli-GPU support would allows to process larger sequences, but the current implementation is not designed for that.

\paragraph{AceZero}
When the camera follows a closed loop around an object (e.g. MipNerf360 \texttt{garden}), AceZero produces crisp trajectories and fair (PSNR \SI{19.89}{dB}, SSIM \SI{0.53}, LPIPS \SI{0.41}).
Performance collapses at outdoor-to-indoor transitions and large scenes with minimal overlap (e.g. LaMAR) cause the scene coordinate regression to diverge.
% Future work might add an adaptive curriculum that injects synthetic views when pose confidence falls below the empirical threshold of $1\,000$.

\paragraph{FlowMap}
Full-batch optical-flow warping makes memory scale linearly with image count (30 GB for 150 frame Tanks And Temples sequence).
Mini-batch optimisation could drop memory by an order of magnitude, enabling evaluation on complete LaMAR runs.

The idea of using depth and known correspondences to compute a camera-induced loss is a promising approach, as it allows to optimise camera parameters in a self-supervised way. 
However, relying on optical flow, limits FlowMap's application to small video sequences. 

\paragraph{Complexity \& resource usage}
Tables in Section \ref{sec:time-and-memory-evaluation} show that GLOMAP is consistently the fastest on ETH3D, MipNerf360 and Tanks And Temples.
VGGSfM is \emph{slower but steady}: once memory is available, it produces relatively accurate trajectories.
AceZero is the slowest on average because it iterates scene-coordinate regression until convergence; however, for trajectories with thousands of frames and wide image overlap, its near-linear regime overtakes COLMAP while maintaining moderate memory usage compared to more memory-intensive alternatives.

\paragraph{Novel-view synthesis quality}
Gaussian Splatting magnifies pose inaccuracies: a small rotation drift can displace splats by tens of pixels. 
Accordingly, COLMAP and GLOMAP, which offer the best pose precision, produce almost identical renders on MipNeRF360; GLOMAP edges ahead in confined interiors (e.g. \texttt{bonsai} 0.92 vs 0.68 SSIM).
VGGSfM's softer geometry lowers PSNR by \SI{\approx3}{dB} on average, whereas FlowMap collapses completely (13.13 dB, SSIM 0.38, LPIPS 0.84).  
AceZero's pinhole model leaves residual distortion that the splatter cannot absorb, explaining its blurry \texttt{room} failure.

% \clearpage
% \paragraph{Limitations of this study}
% \begin{enumerate}
%     \item \textbf{Dataset scope.} We could not test the $\ge10\,000$-image regime advertised by AceZero and GLOMAP because of time constraints.
%     \item \textbf{Intrinsic homogeneity.} Datasets with multiple devices (IMC, ScanNet) were excluded to satisfy AceZero's single-intrinsic assumption; generalising results to heterogeneous rigs thus needs further work.
%     \item \textbf{GPU bias.} Two methods (VGGSfM, FlowMap) exceeded our \SI{32}{GB} VRAM limit on some scenes; their ETH3D scores therefore reflect a favourable subset.
% \end{enumerate}
\chapter{Conclusion}\label{chap:conclusion}
This thesis set out to answer two overarching questions:

\begin{enumerate}
    \item How do recent learning-based SfM pipelines (AceZero, FlowMap, VGGSfM) compare with classical feature-based methods (COLMAP, GLOMAP) in terms of pose accuracy, scalability and downstream novel-view quality ?
    \item What are the practical trade-offs in memory, run-time and completeness when deploying these pipelines on real-world mobile captures ?
\end{enumerate}

To address them, we introduced a unified evaluation protocol that couples rigorous camera-pose metrics (relative/absolute pose errors, AUC) with a task-oriented assessment based on Gaussian-splat novel-view synthesis.  
Benchmarks span controlled indoor scenes (ETH3D), large-scale urban traverses (LaMAR) and high-resolution radiance-field datasets (MipNeRF360, Tanks And Temples).

\subsubsection{Key findings}
\paragraph{GLOMAP excels in accuracy and completeness balance.}
It tops every relative-pose metric on ETH3D and registers $>99\,\%$ of frames, while its median translation error stays below \SI{2}{cm}.  
Rare but catastrophic scale flips (e.g.\ \SI{3.9}{km} outlier in \texttt{courtyard}) slightly inflate mean errors.

\paragraph{COLMAP remains a robust baseline.}
Although it misses more frames than GLOMAP, it delivers the lowest mean translation error (\SI{8.8}{cm}) and the sharpest Gaussian-splat renders in confined interiors.

\paragraph{VGGSfM offers mid-tier accuracy but poor scalability.}
Memory usage rises beyond \SI{30}{GB} for ETH3D scenes and the sliding-window optimiser stalls on long LaMAR sequences.

\paragraph{AceZero is promising yet brittle.}
It shines on circular trajectories (closed-loop MipNeRF360 scenes) but fails during outdoor-to-indoor transitions and registers few images when confidence drops below the $1\,000$ threshold.

\paragraph{FlowMap trails the field.}
Full-batch optical-flow warping drives linear VRAM growth and collapses on wide-baseline imagery, yielding PSNR as low as \SI{13}{dB} on MipNeRF360 .


\subsubsection{Contributions}
\begin{itemize}
    \item A comparative study of five state-of-the-art SfM pipelines across diverse scenes.
    \item An open-source evaluation suite that unifies camera-pose metrics and Gaussian-splat novel-view synthesis.
\end{itemize}

\subsubsection{Limitations}
\begin{enumerate}
    \item \textbf{Dataset scale.} Experiments stopped at 1500 images; claims about $>10\,000$ image performance (especially for AceZero) remain unverified.
    \item \textbf{Single-intrinsic assumption.} To accommodate AceZero, only mono-device sequences were used; heterogeneous-sensor scenarios were left out.
    \item \textbf{Hardware ceiling.} Two methods exceeded the 32 GB VRAM budget, meaning their ETH3D scores stem from down-scaled or truncated runs.
\end{enumerate}

\subsubsection{Directions for future work}
% \begin{itemize}
%     \item \textbf{Learned local features in classical pipelines.} Plugging SuperPoint into COLMAP/GLOMAP via HLoc could boost robustness to blur and lighting changes with minimal engineering effort.
%     \item \textbf{Heterogeneous-sensor evaluation.} Extending the benchmark to mixed-intrinsic datasets would test generalisation beyond the current mono-intrinsic setting.
%     \item \textbf{Full LaMAR dataset.} Rather than treating each LaMAR sessions individually, an evaluation on the full scale dataset would provide a more comprehensive understanding of the methods' performance.
% \end{itemize}
Replacing handcrafted SIFT with learned features (SuperPoint \cite{detone18superpoint} with SuperGlue \cite{sarlin2020superglue} or LighGlue \cite{lindenberger2023lightglue}) inside COLMAP and GLOMAP is a low-hanging fruit: Hloc already provides drop-in wrappers.
% For FlowMap, implementing mini-batch flow warping should unlock much larger sequences.  
% AceZero would benefit from a distortion-aware camera model, and the ability to handle multiple camera intrinsics.
Finally, we suggest extending the benchmark to mixed-intrinsic and large datasets (e.g. Co3D \cite{reizenstein21co3d}, IMC \cite{Jin2020}, ScanNet \cite{dai2017scannet}) to test the generalisation of our findings.

\subsubsection{Closing remarks}
Classical feature-based pipelines, when carefully engineered, still set the bar for metric accuracy and downstream view synthesis.  
Among the methods we evaluated, \textbf{GLOMAP emerges as the sole alternative mature enough to displace COLMAP}: it combines comparable geometric precision with higher registration completeness and markedly faster runtimes.  
Other learning-based approaches remain promising research directions, but they have yet to match this robustness-scalability sweet spot.  
Bridging classical geometry with learned priors, rather than treating them as mutually exclusive paradigms, therefore appears the most fruitful path toward the next generation of structure-from-motion systems.

% \blindmathtrue

% \blinddocument

% \appendix

% \printindex

% \appendix

\bibliographystyle{plain}
\bibliography{src/references}
% \printbibliography

% \ctutemplate{specification.as.chapter}

\end{document}