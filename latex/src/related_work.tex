\chapter{Background and Related Work}\label{chap:related_work}

In this chapter, we provide a comprehensive overview of the foundational concepts and recent advancements relevant to Structure-from-Motion (SfM).
We begin by describing classical SfM techniques, highlighting traditional feature extraction, matching methods, and optimization strategies. 
Subsequently, we review emerging learning-based approaches, focusing on methods that leverage neural networks to enhance reconstruction accuracy, efficiency, and robustness. 
The chapter further delves into detailed discussions of specific state-of-the-art methods, including Glomap, VGGSfm, FlowMap, and Ace Zero, 
clarifying their claimed advantages and limitations compared to classical SfM pipelines like Colmap.

\section{Classical SfM}
SfM approaches can be broadly classified into two categories: incremental and global.

\subsection{Incremental SfM}
Incremental Structure-from-Motion (SfM) follows a sequential approach, where the 3D reconstruction is built progressively by adding one image at a time. 
This method begins with an initial pair of images and expands the reconstruction iteratively by registering new images and refining the existing structure. 
The core steps of incremental SfM include feature extraction and matching, relative camera pose estimation, and bundle adjustment.

\paragraph{Feature Extraction and Matching}
The first step in incremental SfM involves detecting and describing key points in each image. 
Traditional methods use hand-crafted feature descriptors such as SIFT or ORB to extract distinctive image features. 
Once features are detected, feature matching is performed across image pairs to establish correspondences. 
This is typically achieved using nearest-neighbor search with ratio tests to filter out incorrect matches.

\paragraph{Relative Camera Pose Estimation}
Given the matched feature correspondences between image pairs, the next step is to estimate the relative camera poses. 
This is achieved using epipolar geometry constraints, often solved via the five-point or eight-point algorithms within a RANSAC framework to remove outliers. 
The estimated relative camera pose allows for triangulation, reconstructing initial 3D points.

\paragraph{Bundle Adjustment}
As new images are added, the estimated structure and camera parameters can suffers from drift and inaccuracies.
Bundle adjustment is a key step in incremental SfM that refines the camera poses and 3D structure by minimizing the reprojection error.
Non-linear least squares solvers such as Levenberg-Marquardt or Ceres solver are commonly used to achieve this refinement. 
Incremental SfM benefits from high accuracy but suffers from scalability issues as the number of images increases.

\subsection{Global SfM}
In contrast to incremental approaches, global SfM processes all images simultaneously to compute a globally consistent 3D structure.
Instead of reapeting the costly bundle adjustment, global SfM methods estimate camera geometry for all input images at once.

A \emph{view graph} is constructed with all image pairs and their estimated relative poses.

The graph is then used to estimate camera intrinsics, as well as construction of a relative camera pose graph.
The relative camera pose graph is then used to perform \emph{rotation averaging} as well as \emph{translation averaging} \cite{Chatterjee2013, theia-manual, moulon2016openmvg}
Finally, with both global rotations and translations estimated, a global bundle adjustment is performed to refine the camera parameters and 3D point positions, ensuring a consistent and accurate reconstruction of the scene.

This approach allow for more scalable reconstructions.

\section{Learning-based SfM}


\section{Glomap}

Glomap has been presented at the ECCV 2024 conference by \textit{L. Pan et al.} \cite{pan2024glomap}.


\section{VGGSfm}


\section{FlowMap}


\section{Ace Zero}
