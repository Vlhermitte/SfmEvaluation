\chapter{Methodology}\label{chap:methodology}

This chapter describes the evaluation protocol used to assess the performance of the four SfM pipelines: AceZero, GLOMAP, VGGSfm, and FlowMap and compare them with the baseline COLMAP.
We first present the proposed evaluation protocol. Then, we describe the datasets used for evaluation, including their characteristics and the rationale behind their selection.

\section{Proposed evaluation protocol}\label{sec:proposed-evaluation-protocol}

In this section, we present a comprehensive evaluation protocol for assessing the performance of Structure from Motion (SfM) algorithms.
The proposed evaluation protocol consists of four main components: Camera Pose Evaluation, Novel View Synthesis Evaluation, Time performance, and GPU memory usage.
These components are designed to evaluate the accuracy of camera pose estimation and the quality of the reconstructed scene geometry, respectively.

\subsection{Camera Pose Evaluation}\label{sec:camera-pose-evaluation}
To quantitatively assess camera pose accuracy, we compute relative rotations and translations between all possible pairs of estimated camera poses and compare these to the corresponding ground-truth poses.
Additionally, we compare the absolute errors in camera poses between the estimated and ground-truth poses.

\subsubsection{Relative Pose Error}
Given two cameras \( i \) and \( j \), let the relative rotation and translation from camera \( i \) to camera \( j \) be defined as:
\begin{equation}
    R_{ij} = R_j R_i^T
\end{equation}

\begin{equation}
    \vec{t}_{ij} = \vec{t}_i - R_{ij} \vec{t}_j
\end{equation}

where \( R_i \) and \( R_j \) are the rotation matrices of cameras \( i \) and \( j \) respectively, and \( \vec{t}_i \) and \( \vec{t}_j \) are their respective translation vectors.

\paragraph{Relative Translation Error (RTE)}
The Relative Translation Error (RTE) measures the angular difference between the estimated and ground-truth camera relative translations and is computed as:

\begin{equation}
    \text{RTE}_{ij} = \cos^{-1}\left(\frac{\vec{t_{ij}}^{\text{gt}} \cdot \vec{t_{ij}}^{\text{est}}}{\lVert \vec{t_{ij}}^{\text{gt}} \rVert \lVert \vec{t_{ij}}^{\text{est}} \rVert}\right)
\end{equation}

\paragraph{Relative Rotation Error (RRE)}
The Relative Rotation Error (RRE) quantifies the angular difference between estimated and ground-truth relative rotations. It is defined as:

\begin{equation}
    \text{RRE}_{ij} = \cos^{-1}\left(\frac{\text{trace}(R_{ij}^{\text{gt}} (R_{ij}^{\text{est}})^T) - 1}{2}\right)
\end{equation}

if a pose is missing, we apply a high penalty of 180 degrees to both the RTE and RRE.

\paragraph{Area Under the Curve (AUC)}
To summarize the performance of the SfM algorithms, we report in Section \ref{sec:camera-pose-estimation-results} the Area Under the Curve (AUC) for the cumulative distribution of angular errors at different thresholds. 
This provides a comprehensive assessment of the algorithm's accuracy across varying levels of tolerance.

Following the evaluation protocol of \emph{Wang et al.} \cite{wang2023vggsfm}
For every camera pair \((i,j)\) we first condense the two angular errors into a single value by taking their maximum
\begin{equation}
    e_{ij} = \max\!\bigl(\text{RRE}_{ij},\,\text{RTE}_{ij}\bigr)
\end{equation}

so that a large error in either rotation or translation is directly reflected in \(e_{ij}\).

Let the set of bins be \(\mathcal{B} = \{0^{\circ}, 1^{\circ}, 2^{\circ}, \ldots, \tau_{\max}\}\) with \(\tau_{\max}=30^{\circ}\).

We compute the normalized histogram of errors \(h\) as:
\begin{equation}
    h_k = \frac{1}{N} \sum_{i,j} \mathbb{I}(e_{ij} \leq k)
\end{equation}
where \(N\) is the number of camera pairs and \(\mathbb{I}\) is the indicator function.
The AUC is then computed as mean of the cumulative sum of the histogram:
\begin{equation}
    \text{AUC} = \frac{1}{|\mathcal{B}|} \sum_{k=0}^{\tau_{\max}} h_k
\end{equation}
The AUC value ranges from 0 to 1, where 1 indicates perfect accuracy and 0 indicates no accuracy.

Similarly, we also report in \ref{sec:camera-pose-estimation-results} the area under the curve for the relative translation error (RTE) and relative rotation error (RRE) separately.
The AUC values for RTE and RRE are computed in the same way as for the overall error \(e_{ij}\), but using only the respective errors.


\subsubsection{Absolute Pose Error}
The Absolute Pose Error measures the absolute difference between the estimated and ground-truth camera poses.

In order to compute it, we first need to align the models. Indeed, the estimated reconstruction might be in a different coordinate system than the ground-truth reconstruction.

We align the estimated reconstruction with the ground-truth reconstruction using COLMAP \textit{model aligner}.

After alignment, the error in pose can be decomposed into two parts: the \textbf{rotation error} and the \textbf{camera position (translation) error}.

The camera position in the world coordinate system is given by:
\begin{equation}
    \vec{C} = -R^\top \cdot \vec{t}
\end{equation}
where the rotation matrix $R$ is the world orientation in the camera coordinate frame and the translation vector $\vec{t}$ is the world origin in the camera coordinate frame.

To compute the camera position error, we compare the estimated camera center $\vec{C}_{\text{est}}$ to the ground-truth camera center $\vec{C}_{\text{gt}}$ using the Euclidean distance:
\begin{equation}
    \vec{C}_{\text{error}} = \left\| \vec{C}_{\text{est}} - \vec{C}_{\text{gt}} \right\|
\end{equation}

The rotation error is computed using the relative rotation matrix:
\begin{equation}
    R_{\text{error}} = R_{\text{est}} R_{\text{gt}}^\top
\end{equation}

The angle of rotation $\alpha$ between the estimated and ground-truth rotations is then calculated as:

\begin{equation}
    \alpha = \cos^{-1}\left( \frac{\text{trace}(R_{\text{error}}) - 1}{2} \right)
\end{equation}

This gives the angular difference in radians. The result can be converted to degrees if needed.


\subsection{Novel View Synthesis Evaluation}

To evaluate the quality and accuracy of the reconstructed scene geometry, we employ Novel View Synthesis using Gaussian Splatting.
Specifically, the \texttt{GSplat} \cite{ye2024gsplatopensourcelibrarygaussian} framework is utilized to render synthetic images from novel camera viewpoints not included in the original set of images used for reconstruction. 
GSplat is an open-source library that implements Gaussian Splatting for real-time rendering of neural radiance fields (NeRFs) and is insipred by the original work on Gaussian Splatting \cite{kerbl20233dgaussiansplattingrealtime} but was improved to be faster and more memory efficient.

3D Gaussian Splatting methods have recently demonstrated state-of-the-art fidelity and real-time performance, motivating their adoption in place of Neural Radiance Fields.

The quality of these synthetic views is quantitatively assessed using standard metrics, including Peak Signal-to-Noise Ratio (PSNR), Structural Similarity Index Measure (SSIM), and Learned Perceptual Image Patch Similarity (LPIPS). 

\begin{itemize}
    \item \textbf{PSNR}: Measures the ratio of peak signal power to noise power in the image, expressed in decibels (dB). Higher values indicate lower distortion.
    \item \textbf{SSIM}: Evaluates perceptual similarity by comparing luminance, contrast, and structure between images. A higher SSIM score implies a closer match to the reference image.
    \item \textbf{LPIPS}: Uses deep learning-based feature comparisons to assess perceptual similarity. Lower LPIPS scores indicate greater visual fidelity. AlexNet \cite{krizhevsky2012imagenet} is used as feature extractors for LPIPS.
\end{itemize}

Higher PSNR and SSIM scores, along with lower LPIPS scores, indicate superior reconstruction fidelity and scene representation accuracy.

Novel View Synthesis is a convenient way to evaluate the quality of the reconstructed scene geometry, as it doesn't require any additional data or ground-truth information.
Only the set of images used for reconstruction is needed, and the evaluation can be performed on any scene.

\subsubsection{Split Training and Evaluation Sets}\label{sec:split_sets}

We first sort the images according to their acquisition order and assign every
\emph{eighth} frame to the evaluation split, resulting in a deterministic $90\%$ training and $10\%$ evaluation split.
Formally, let $\mathcal{I}=\{I_1,\dots,I_N\}$ be the set of frames that were successfully SfM-registered by a given pipeline together with their camera poses
$\{\mathbf{T}_1,\dots,\mathbf{T}_N\}$. \\
\\
The training set $\mathcal{T}$ and evaluation set $\mathcal{E}$ are defined as:
\begin{equation}
  \mathcal{T} = \{I_k, \mathbf{T}_k\}_{k \in \mathcal{I}, k \mod 8 \neq 0}\,
\end{equation}
\begin{equation}
  \mathcal{E} = \{I_k, \mathbf{T}_k\}_{k \in \mathcal{I}, k \mod 8 = 0}\,
\end{equation}
The training set $\mathcal{T}$ is used to train the Gaussian Splatting model, while the evaluation set $\mathcal{E}$ is used to evaluate the performance of the SfM pipeline.

\paragraph{Handling missing registrations.}
In practice, the four tested SfM pipelines do not register exactly the same subset of images.  
A naïve stride-8 split would therefore evaluate each pipeline on a slightly different set of viewpoints, making the numerical comparison unfair.
Indeed, if a methods does not register a smaller amount of images compare to other SfM methods, this would results in a smaller subset $\mathcal{E}$ and thus the mean PSNR, SSIM and LPIPS would be the results of fewer images evaluated.

To guarantee that every algorithm is scored from identical viewpoints, we fill the gaps in $\mathcal{E}$: 
if an image $I_k$ is missing for a particular pipeline, we use the RGB content of $I_k$ together with the pose of the closest registered frame,
\begin{equation}
  \mathbf{T}_{\operatorname*{arg\,min}_{j\in\mathcal{I}} |j-k| }\,
\end{equation}

Applying this rule implies that some ground truth images will be compared with reconstructed images from a different pose that is the closest. 
Consequently, the resulting PSNR, SSIM, and LPIPS values will be affected by this heuristic.

However, we believe that this approach strikes a good balance between fairness and evaluation consistency. 
It also serves as an effective way to penalize the absence of poses.

Since this substitution is applied symmetrically to all pipelines, the visual content is shared, and the evaluation remains fair.

\paragraph{No pose imputation during training.}
For the Gaussian-Splatting optimisation itself we deliberately do not impute the missing poses: inaccurate or duplicated camera parameters act as label noise and noticeably hinder convergence. 
Consequently, each pipeline is trained on its own (possibly reduced) set $\mathcal{T}$, but all are evaluated on the common set $\mathcal{E}$. 
Empirically, we found that this strategy offers the best trade-off between training stability and the comparability of the final metrics.


\subsection{Time and Memory Performance}
In addition to reconstruction quality, we report the computational performance of the evaluated SfM algorithms in terms of both time and memory efficiency. 
These aspects are particularly important when considering deployment in real-world applications, especially those requiring real-time processing or operating on large-scale datasets.

Time performance is measured as the total processing time needed to reconstruct a scene from the input images, which includes camera pose estimation and sparse point cloud generation. 
This metric reflects the overall computational efficiency of the algorithm.

Memory performance is assessed by recording the peak GPU memory usage during reconstruction. Memory efficiency is a critical factor for scalability, as not all users have access to high-end GPUs with large memory capacities. 
Such GPUs are not only expensive, but also increasingly difficult to obtain due to sustained high demand over the past few years in both academic and industrial settings. 
Efficient memory usage therefore enables broader accessibility and allows SfM algorithms to run on more modest hardware, facilitating adoption in resource-constrained environments.


\section{Dataset Selection}\label{sec:dataset-selection}

In this section, we provide an overview of the datasets used for evaluation and the rationale behind their selection.
The chosen datasets span a diverse range of scenarios, including both indoor and outdoor environments, varying lighting conditions, and different types of scene geometry and motion trajectories.
Some datasets feature long-range trajectories across large environments (e.g., urban or campus-scale), while others involve constrained camera motion around a central object or scene.

It is important to note that AceZero assumes consistent camera intrinsics across all images, which limits its applicability to datasets where intrinsics vary between frames.
Therefore, we excluded datasets such as IMC \cite{Jin2020}, which include multiple devices or cameras with different intrinsic parameters.

\subsection{ETH3D Stereo}
The ETH3D Stereo dataset \cite{schoeps2017cvpr} is a widely used benchmark for evaluating Structure-from-Motion (SfM) and stereo reconstruction algorithms. 
We selected 13 scenes from the high-resolution subset, which includes both indoor and outdoor environments. 
The camera captures images of either a large scene while retraining a relatively small number of images (e.g., 10 to 70 images).
All images within a scene are captured using the same DSLR camera and lens, ensuring consistent camera intrinsics throughout. 
The dataset provide ground-truth camera poses, but it does include high-quality 3D scan models of the scenes, which can be used for evaluating the accuracy of the reconstructed 3D points.

\subsection{LaMAR}
The LaMAR dataset \cite{sarlin2022lamar} is a large-scale benchmark designed for evaluating localization and mapping algorithms in augmented reality scenarios.
It includes a wide range of challenging environments with both indoor and outdoor scenes, complex geometry, and varied lighting conditions.
We use three large-scale scenes, each divided into multiple sessions. Each session contains a sequence of images captured with the same iOS device, ensuring consistent intrinsics.
While LaMAR also includes data from the Microsoft HoloLens, we only use the iOS device images to comply with the single-intrinsic constraint of AceZero. 

Similarity to ETH3D, the LaMAR dataset captures long-range trajectories, where the camera moves around a large scene, such as a building or a campus. 
However, it provide much more images, allowing to both highlights the performance of the SfM algorithm and their scalability.
The dataset provides high-quality ground-truth camera poses, making it suitable for evaluating camera pose estimation and robustness in real-world, dynamic environments.

\subsection{MipNeRF360}
The MipNeRF360 dataset \cite{barron2022mipnerf360} is designed for evaluating neural radiance field (NeRF) models in unbounded, 360-degree scenes.
It consists of several scenes where the camera orbits around a central object or region of interest, capturing a dense set of images from multiple viewpoints.
All scenes are captured under controlled lighting with consistent camera intrinsics, making it compatible with AceZero. 
This dataset is especially valuable for testing systems on closed-loop trajectories with complex object geometry and view-dependent effects.
Ground-truth camera poses are provided, enabling detailed evaluation of pose accuracy and consistency.

\subsection{Tanks and Temples}
The Tanks and Temples dataset \cite{Knapitsch2017} is a standard benchmark for evaluating multi-view stereo and 3D reconstruction techniques.
It consists of high-resolution image sequences of complex scenes such as statues, monuments, and architectural structures. 
Each sequence captures the camera rotating around a central subject, with sufficient overlap and varying viewpoints.
Scenes are recorded under natural outdoor lighting, introducing realistic challenges such as shadows and specularities.
All images within a scene share the same camera intrinsics, and accurate ground-truth 3D reconstructions are available, allowing both qualitative and quantitative assessments of SfM pipelines.